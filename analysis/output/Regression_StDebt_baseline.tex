\documentclass{article}
\usepackage{booktabs}
\usepackage{tabularx}
\usepackage[margin=1in]{geometry}
\begin{document}

\begin{table}[tbp] \centering
\newcolumntype{C}{>{\centering\arraybackslash}X}

\caption{Effect of Cash Reserves on Short Term Debt Issuance}
\label{tab:Regression_StDebt}
\begin{tabularx}{\linewidth}{lCCCC}

\toprule
&{(1)}&{(2)}&{(3)}&{(4)} \tabularnewline \midrule
{}&{}&{}&{}&{} \tabularnewline
\midrule \addlinespace[\belowrulesep]
\textbf{Panel A: OLS Estimates}&&&& \tabularnewline
\midrule Cash Reserves (\% DR)&--0.1512***&--0.0350&0.0609*&0.0740* \tabularnewline
&(0.0298)&(0.0327)&(0.0349)&(0.0379) \tabularnewline
\textbf{Panel B: 2SLS IV Estimates}&&&& \tabularnewline
\midrule Cash Reserves (\% DR)&0.1616&0.1879&0.1911*&0.1933* \tabularnewline
&(0.1508)&(0.1196)&(0.1022)&(0.1029) \tabularnewline
First Stage: Budget Error $\hat{\beta}$&1.5493***&1.6448***&1.6842***&1.6262*** \tabularnewline
&(0.5574)&(0.4558)&(0.4254)&(0.4134) \tabularnewline
Cragg-Donald Wald F statistic&7.7256&13.0206&15.6723&15.4714 \tabularnewline
\midrule Mean of Dep Var&0.0523&0.0523&0.0523&0.0523 \tabularnewline
Observations&626&626&626&626 \tabularnewline
Controls&No&Yes&No&Yes \tabularnewline
State FE&No&No&Yes&Yes \tabularnewline
Time FE&Yes&Yes&Yes&Yes \tabularnewline
\bottomrule \addlinespace[\belowrulesep]

\end{tabularx}
\begin{flushleft}
\footnotesize Notes: Panel A shows the results of the linear regression model across several specifications. Panel B displays the results of the 2SLS regression where the budget error instruments cash reserves. All the dependent, independent, and instrumental variables are expressed as a percentage of each state's average discretionary revenues (DR) from 2009-2016. Standard errors clustered at the state level. A */**/*** indicates significance at the 10/5/1\% levels.
\end{flushleft}
\end{table}
\end{document}
