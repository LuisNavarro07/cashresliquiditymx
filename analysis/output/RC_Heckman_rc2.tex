\documentclass{article}
\usepackage{booktabs}
\usepackage{tabularx}
\usepackage[margin=1in]{geometry}
\begin{document}

\begin{table}[tbp] \centering
\newcolumntype{C}{>{\centering\arraybackslash}X}

\caption{Heckman Selection Model: Short Term Borrowing and Cash Reserves}
\label{tab:Regression_StDebt}
\begin{tabularx}{\linewidth}{lCCCCC}

\toprule
&{(1)}&{(2)}&{(3)}&{(4)}&{(5)} \tabularnewline \midrule
{}&{}&{}&{}&{}&{} \tabularnewline
\midrule \addlinespace[\belowrulesep]
\textbf{Panel A: Second Stage (Outcome Model)}&&&&& \tabularnewline
\midrule Cash Reserves (\% DR)&--0.1674***&--0.1001*&0.0238&0.0342&--0.1001* \tabularnewline
&(0.0535)&(0.0558)&(0.0606)&(0.1694)&(0.0558) \tabularnewline
\midrule \textbf{Panel B: First Stage (Selection Model)}&&&&& \tabularnewline
Budget Error (\% DR)&--2.1133&--2.1133&--2.1133&--2.1133&--2.1133 \tabularnewline
&(3.2889)&(3.2889)&(3.2889)&(3.2889)&(3.2889) \tabularnewline
Mean of Dep Var&0.0520&0.0520&0.0520&0.0520&0.0520 \tabularnewline
\midrule Observations&629&629&629&629&629 \tabularnewline
Controls&No&Yes&No&Yes&Yes \tabularnewline
State FE&No&No&Yes&Yes&No \tabularnewline
Time FE&Yes&Yes&Yes&Yes&Yes \tabularnewline
\bottomrule \addlinespace[\belowrulesep]

\end{tabularx}
\begin{flushleft}
\footnotesize Notes: Panel A shows the results from the second stage regression. Panel B shows displays the results of the instrument used for the selection model. Estimation is done using Heckman's (1979) two-step efficient estimates of parameters and standard errors. Results in Column (5) replicate the econometric specification at \citep{suDoesFinancialSlack2018}. All the dependent, independent, and instrumental variables are expressed as a percentage of each state's average discretionary revenues (DR) from 2009-2016. Standard errors clustered at the state level A */**/*** indicate significance at the 10/5/1\% levels.
\end{flushleft}
\end{table}
\end{document}
