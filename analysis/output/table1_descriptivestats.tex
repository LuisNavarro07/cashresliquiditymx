\begin{table}[tbp] \centering
\newcolumntype{C}{>{\centering\arraybackslash}X}

\caption{Descriptive Statistics}
\label{tab:table1_descriptivestats}
\begin{tabularx}{\linewidth}{lCCCCCC}

\toprule
&{(1)}&{(2)}&{(3)}&{(4)}&{(5)}&{(6)} \tabularnewline \midrule
{}&{Mean}&{SD}&{P25}&{P50}&{P75}&{N} \tabularnewline
\midrule \addlinespace[\belowrulesep]
\midrule Short-Term Debt (\% DR)&0.0523&0.0641&0.0000&0.0240&0.0965&630 \tabularnewline
Cash Reserves (\% DR)&0.2255&0.1544&0.1139&0.1861&0.3060&626 \tabularnewline
FGP Budget Error (\% DR)&-.0043&0.0229&-.0181&-.0041&0.0067&630 \tabularnewline
Net Operating Balance (\% DR)&-.0605&0.1246&-.0833&-.0261&0.0018&630 \tabularnewline
Credit Rating&2.2428&0.7108&2.0000&2.0000&3.0000&630 \tabularnewline
\% FGP Securing LT debt&0.5352&0.2161&0.3316&0.5476&0.75&630 \tabularnewline
Current Expenditure (\% Total Expenditure)&0.7382&0.0591&0.7133&0.7531&0.7775&630 \tabularnewline
Discretionary Revenue (\% Total Revenue)&0.4747&0.0786&0.4168&0.4714&0.5382&630 \tabularnewline
Long Term Debt (\% Total Debt)&0.6683&0.5118&0.2833&0.5652&0.8573&630 \tabularnewline
\bottomrule \addlinespace[\belowrulesep]

\end{tabularx}
\begin{flushleft}
\footnotesize Notes: This panel show the descriptive statistics of the main variables used for the analysis. The first two columns show the sample mean and standard deviation. P25, P50 and P75 show the 25, 50 and 75 percentiles, respectively. Credit rating is coded such that a higher number is associated with a higher credit rating. Considering the distribution of ratings I grouped them in 3 categories AAA,AA = 1, A = 2, and BBB,BB,NR = 3. Short-Term borrowing, cash reserves, FGP budget error, and Net Operating Balance are expressed as a percentage of the average discretionary revenues (DR) observed between 2009 and 2016. That is, outside the analysis period to avoid endogeneity concerns. Net operating balance, current expenditures, and discretionary revenues correspond to one year lagged measures. 
\end{flushleft}
\end{table}
